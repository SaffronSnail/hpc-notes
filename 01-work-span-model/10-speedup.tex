\documentclass{standalone}

\title{Speedup}
\date{2017-07-09}
\author{Bryan Ferris}

\usepackage{amsmath}
\usepackage[subpreambles=true]{standalone}

\begin{document}

% \ifstandalone
% \maketitle
% \fi

\section{Speedup}

\subsection{Definitions}
\begin{align*}
	Sp &= \text{Speedup}\\
	T_1 &= \text{Best linear time}\\
	T_p &= \text{Time of a specific parallel algorithm}\\
	Sp &= \frac{T_1}{T_p}
\end{align*}

Speedup is how many times faster $T_p$ is than $T_1$:
\begin{align*}
	Sp \times T_p = \frac{T_1 \times T_p}{T_p}\\
	Sp \times T_p = T_1
\end{align*}

\subsection{Ideal Speedup}
Ideal speedup is achieved when $T_p$ is $P$ times faster than $T_1$. That is, the algorithm scales linearly with $P$.

\subsubsection{Achieving Ideal Speedup}
Consider the following:
\begin{align*}
  Sp(n) &= \frac{T_1}{T_p}\\
  Sp(n) &= \frac{W}{T_p} \geq \frac{W}{\frac{W_p - D}{P} + D}\\
  Sp(n) &= \frac{W}{T_p} \geq \frac{P}{\frac{W_p}{W} + \frac{P - 1}{\frac{W}{D}}}
\end{align*}

\end{document}

