\documentclass{standalone}

\lstdefinelanguage{myLang}
{
	% list of keywords
	morekywords={
		for,
		parfor
	}
	sensitive=true, % keywords are case sensitive
	morecomment=[l]{//}, % l is for line comment
	morecomment=[s]{/*}{*/}, % s is for start and end delimiter
	morestring=[b]" % defines that strings are enclosed in double quotes
}



\usepackage{amsmath}
\usepackage{lastpage}
\usepackage[subpreambles=true]{standalone}

\title{The Work-Span Model}
\date{2017/08/09}
\author{Bryan Ferris}

\begin{document}
\ifstandalone
\maketitle
\pagenumbering{gobble}

\newpage
\pagenumbering{arabic}
\fi

\section{Spawn and Sync}
To spawn a function is to run it in a new thread. To sync is to wait for certain
threads to finish executing before continuing (which threads you are waiting for
should be clear from the context of the sync).

\section{Work and Span}
Work, $W$, refers to the number of nodes in the DAG, while Span, $D$, refers to
the longest path length from the beginning to the end of the DAG.\\

In most situations, ideal $W$ will be $O(n)$. While it can be common to have
even better performance in specific sitations (such as searching algorithms in
sorted data structures), most of the time when you do work there is an operation
that you need to perform on each element. In these cases, better than $O(n)$
time is unrealistic.\\

Ideal $D$, however, is normally $O(log^k(n))$ ($polylogrithmic$).

\subsection{Speedup}

\subsubsection{Definitions}
\begin{align*}
	Sp &= \text{Speedup}\\
	T_1 &= \text{Best linear time}\\
	T_p &= \text{Time of a parallel algorithm using p processors}\\
	Sp &= \frac{T_1}{T_p}
\end{align*}

Speedup is how many times faster $T_p$ is than $T_1$:
\begin{align*}
	Sp \times T_p = \frac{T_1 \times T_p}{T_p}\\
	Sp \times T_p = T_1
\end{align*}

\subsubsection{Ideal Speedup}
Ideal speedup is achieved when $T_p$ is $P$ times faster than $T_1$. That is, the algorithm scales linearly with $P$.

\subsubsection{Achieving Ideal Speedup}
Consider the following:
\begin{align*}
  Sp(n) &= \frac{T_1}{T_p}\\
  Sp(n) &= \frac{W}{T_p} \geq \frac{W}{\frac{W_p - D}{P} + D}\\
  Sp(n) &= \frac{W}{T_p} \geq \frac{P}{\frac{W_p}{W} + \frac{P - 1}{\frac{W}{D}}}
\end{align*}

\subsection{Work and Span Laws}
\subsubsection{Average Available Parallelism}
This tells us how much work is available per critical path node, on average. We
use this to determine k
\[\frac{W}{D}\]
The idea here is that we are figuring out how much work is available
\textit{per critical path node}. If the dependencies in the DAG are spread out
relatively evenly amongst critical path nodes then this figure is very useful:
it tells us how many processors we can keep busy per critical path node.
However, the critical path may not always have dependencies distributed evenly,
meaning that more or less processors could be more efficient overall.

\subsubsection{Span Law}
\[T_p \geq D\]

\subsubsection{Work Law}
\[T_p \geq \lceil \frac{W}{P} \rceil\]

\subsubsection{Brent's Theorem}
Brent's Theorem is derived as follows:

\begin{enumerate}
  \item \textbf{Divide the DAG into D phases}. Nodes may only depend on nodes
    from previous phases (NOT the same phase). Each phase should contain exactly
    one critical path node. A phase is identified by the number of the critical
    path node (where the number is their natural numbering from start to end).
  \item \begin{align*}
    T_p &= \sum_{k=1}^D \lceil\frac{W_k}{P}\rceil\\
    T_p &= \sum_{k=1}^D \lfloor\frac{W_k - 1}{P}\rfloor + 1\\
    T_p &\leq \sum_{k=1}^D \frac{W_k - 1}{P} + 1\\
    T_p &\leq \frac{W - D}{P} + D
  \end{align*}
\end{enumerate}

\subsection{Parallel For}
Parallel for, or parfor, is shorthand for writing a loop where each iteration of
the loop may be performed in parallel. That is, the following two code segments
are identical:

\begin{lstlisting}
  parfor i in 1 to n do
    foo(i)
\end{lstlisting}

\begin{lstlisting}
  fn ParForT(foo, a, b)
    let n = b - a + 1
    if n = 1 then foo(a)
    else
      let m = a + floor(n/2)
      spawn ParForT(foo, a, m - 1)
      ParForT(foo, m, b)
      sync
  ParForT(foo, 1, n)
\end{lstlisting}

Note the recursive implementation of ParForT.
While it would be simpler to write:

\begin{lstlisting}
  fn ParForT(foo, a, b)
  for i in a to b do
    spawn foo(i)
  sync()
\end{lstlisting}

This is suboptimal. Since the spawns are executed linarly, the minimum size of
the critical path is $b - a$. Writing parfor as a recursive function allows the
spawns to happen in parallel, rather than only the foo function happening in
parallel, bringing the minimum critical path size down to $log(n)$. This is
particularly important when performing fast operations on large data sets.

\end{document}

